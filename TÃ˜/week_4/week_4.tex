\documentclass{article}
\usepackage{amsmath}
\author{Valdemar Emil Otte Petersen}
\title{TØ uge X}
\begin{document}

\begingroup

\centering 
{\LARGE Theory of Measure and Integration - TØ}\\ % Title
\vspace*{1\baselineskip}
\scshape
Week 4\\ 
Valdemar Emil Otte Petersen\\ % Name
{\small Aarhus Universitet}\\ 
{\small \today}

\endgroup

\section{Opgaver}
{\LARGE Question 1.18}\\

\textbf{Opgavebeskrivelse:}\\
text

\vspace{15px}
\textbf{Løsning:}\\
text


\vspace{35px}
{\LARGE Question 1.21}\\

\textbf{Opgavebeskrivelse:}\\
text

\vspace{15px}
\textbf{Løsning:}\\
text

\vspace{35px}
{\LARGE Question 1.22}\\

\textbf{Opgavebeskrivelse:}\\
text

\vspace{15px}
\textbf{Løsning:}\\
text

\vspace{35px}
{\LARGE Question 4.1}\\

\textbf{Opgavebeskrivelse:}\\
text

\vspace{15px}
\textbf{Løsning:}\\
(a)

(b)

\vspace{35px}
{\LARGE Question 4.3}\\

\textbf{Opgavebeskrivelse:}\\
text

\vspace{15px}
\textbf{Løsning:}\\
text

\vspace{35px}
{\LARGE Question 4.4}\\

\textbf{Opgavebeskrivelse:}\\
Lad $(X, \mathcal{E})$ være et måleligt rum, lad $f$ og $g$ være funktioner fra $\mathcal{M(E)}$, og lad $A$ være en mængde fra $\mathcal{E}$.
Vis da, at funktionen $h: X \rightarrow R$ givet ved

\begin{equation}
    h(x) = 
    \begin{cases}
        f(x), x \in A\\
        g(x), x \in A^c
    \end{cases}
\end{equation}

igen er et element i $\mathcal{M(E)}$

\vspace{15px}
\textbf{Løsning:}\\
Vi vil gerne bruge Sætning 4.4.3 (Tuborg resultatet).

For at bruge Tuborg resultatet, så skal der gælde for alle $j$, at $f_j$ er $\mathcal{E} j-\mathcal{F}$-målelige.

Først bemærker vi, at $A \cup A^C = X$.

Vi bemærker også, at følge bemærkning 4.4.2 (3), at vi kan betragte restriktionen $f|_{A} : A \rightarrow R$ givet ved $f|_{A}(x)= f(x), (x\in A)$.
Idet $f|_{A} \cdot i_{A}$, så følger det fra Sætning 4.1.6(v), at $f|_{A}$ er $\mathcal{E}_{A}$-$\mathcal{F}$-målelig.
Tilsvarende kan gøres for $g|_{A^c}$

Dermed vil vi nu kunne bruge Tuborg resultatet, og sige at afbildningen er $\mathcal{E}$-$\mathcal{F}$-målelig og dermed igen et element i $\mathcal{M(E)}$

\vspace{35px}
{\LARGE Question 4.6}\\

\textbf{Opgavebeskrivelse:}\\
text

\vspace{15px}
\textbf{Løsning:}\\
text


\vspace{35px}
{\LARGE Question 4.7}\\

\textbf{Opgavebeskrivelse:}\\
text

\vspace{15px}
\textbf{Løsning:}\\
text

\end{document}